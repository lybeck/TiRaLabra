\documentclass[10pt,a4paper]{article}
\usepackage[utf8]{inputenc}
\usepackage[finnish]{babel}
\usepackage{amsmath}
\usepackage{amsfonts}
\usepackage{amssymb}

\usepackage{verbatim}

\usepackage{setspace}
\onehalfspacing

\newcommand{\N}{\ensuremath{\mathbb{N}}}
\newcommand{\Z}{\ensuremath{\mathbb{Z}}}
\newcommand{\Q}{\ensuremath{\mathbb{Q}}}
\newcommand{\R}{\ensuremath{\mathbb{R}}}
\newcommand{\C}{\ensuremath{\mathbb{C}}}

\newcommand{\pseudo}[1]{{\small\verbatiminput{pseudocode/#1.txt}}}


\author{Lasse Lybeck}
\title{Toteutusdokumentti}


\begin{document}

\maketitle

\section{Ohjelman yleisrakenne}


\section{Saavutetut aika- ja tilavaativuudet}

\subsection{Yhteenlasku}

\pseudo{add}

Kahden $m\times n$ matriisin yhteenlaskun aikavaatimukseksi saadaan selvästi $O(mn)$.
Jos kyseessä on kahden $n\times n$ neliömatriisin yhteenlasku saadaan siis aikavaatimukseksi $O(n^2)$.

Vähennyslasku on täysin ekvivalentti yhteenlaskun kanssa, joten vaativuudet ovat samat.

\subsection{Kertolasku}

\subsubsection{Skalaarilla kertominen}

\pseudo{scale}

$m\times n$ matriisin skalaarilla kertominen on selvästi aikavaativuudeltaan $O(mn)$. $n\times n$ neliömatriisille
tämä siis on $O(n^2)$.

\subsubsection{Matriisikertolasku}

\pseudo{mul}

Matriisikertolaskussa ensimmäisen matriisin sarakkeiden määrä tulee olla sama kuin toisen matriisin rivien määrä.
Matriisien $A\in \R^{m\times n}$ ja $B\in \R^{n\times p}$ tuloksena saadaan siis matriisi $AB\in \R^{m\times p}$.
Naiivia algoritmia (yllä) seuraamalla aikavaativuudeksi saadaan $O(mnp)$, joka neliömatriisien $A,B\in \R^{n\times n}$
tapauksessa on $O(n^3)$.

Kappaleessa \ref{sec:vertailu} sivulla \pageref{sec:vertailu} verrataan naiivia algoritmia Strassenin asymptoottisesti
nopeampaan algoritmiin.

\subsection{Potenssiin korottaminen}

\pseudo{pow}

\subsection{LU-hajotelma ja determinantti}

\pseudo{LU}

\subsection{Käänteismatriisin laskeminen}

\pseudo{inv}


\section{Suorituskyky- ja O-analyysivertailu}\label{sec:vertailu}


\section{Työn mahdolliset puutteet ja parannusehdotukset}


\section{Lähteet}


\end{document}